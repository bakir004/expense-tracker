\subsection{Opis API-ja}

Expense Tracker API je RESTful API izgrađen pomoću ASP.NET Core-a za upravljanje korisnicima, kategorijama, transakcijama i grupama transakcija. API prati Onion Architecture obrazac sa jasnom separacijom između Domain, Application, Infrastructure i WebApi slojeva.

\textbf{Osnovne karakteristike:}
\begin{itemize}
    \item \textbf{Base URL}: \texttt{/api/v1}
    \item \textbf{Swagger UI}: \texttt{/swagger} (interaktivna dokumentacija)
    \item \textbf{Format podataka}: JSON
    \item \textbf{Verzija}: v1 (sve rute su verzionisane)
    \item \textbf{Error handling}: RFC 9110 Problem Details format
    \item \textbf{Arhitektura}: Onion Architecture sa Repository pattern i ErrorOr pattern za funkcionalno rukovanje greškama
\end{itemize}

\textbf{Link na dokumentaciju}: \url{https://github.com/bakir004/expense-tracker}

\subsection{REST principi}

API prati REST (Representational State Transfer) principe:

\begin{itemize}
    \item \textbf{Stateless komunikacija}: Svaki HTTP zahtjev sadrži sve potrebne informacije. Server ne čuva stanje između zahtjeva.
    
    \item \textbf{HTTP metode}: API koristi standardne HTTP metode:
    \begin{itemize}
        \item \texttt{GET} - za dohvaćanje resursa (npr. \texttt{GET /api/v1/users})
        \item \texttt{POST} - za kreiranje novih resursa (npr. \texttt{POST /api/v1/categories})
        \item \texttt{PUT} - za ažuriranje postojećih resursa (npr. \texttt{PUT /api/v1/transactions/\{id\}})
        \item \texttt{DELETE} - za brisanje resursa (npr. \texttt{DELETE /api/v1/categories/\{id\}})
    \end{itemize}
    
    \item \textbf{Resource-based URL-ovi}: URL-ovi predstavljaju resurse, ne akcije:
    \begin{itemize}
        \item \texttt{/api/v1/users} - kolekcija korisnika
        \item \texttt{/api/v1/users/\{id\}} - pojedinačni korisnik
        \item \texttt{/api/v1/transactions/user/\{userId\}} - transakcije određenog korisnika
    \end{itemize}
    
    \item \textbf{JSON format}: Svi podaci se razmjenjuju u JSON formatu. Request body i Response body su u JSON-u.
    
    \item \textbf{HTTP status kodovi}: API koristi standardne HTTP status kodove:
    \begin{itemize}
        \item \texttt{200 OK} - uspješan zahtjev
        \item \texttt{201 Created} - resurs uspješno kreiran
        \item \texttt{204 No Content} - uspješno brisanje (bez body-ja)
        \item \texttt{400 Bad Request} - validacione greške
        \item \texttt{404 Not Found} - resurs nije pronađen
        \item \texttt{409 Conflict} - konflikt (npr. duplikat imena kategorije)
        \item \texttt{422 Unprocessable Entity} - semantičke validacione greške
        \item \texttt{500 Internal Server Error} - neočekivane greške
    \end{itemize}
\end{itemize}

\subsection{Klase u bazi podataka}

API koristi PostgreSQL bazu podataka sa sljedećim glavnim entitetima:

\subsubsection{User (Korisnik)}
Entitet koji predstavlja korisnika sistema.
\begin{itemize}
    \item \texttt{Id}: \texttt{int} - jedinstveni identifikator
    \item \texttt{Name}: \texttt{string} - puno ime korisnika (1-100 karaktera)
    \item \texttt{Email}: \texttt{string} - email adresa (jedinstvena, validan email format)
    \item \texttt{PasswordHash}: \texttt{string} - hash lozinke (ne vraća se u API odgovorima)
    \item \texttt{InitialBalance}: \texttt{decimal} - početni balans korisnika
    \item \texttt{CreatedAt}: \texttt{DateTime} - vrijeme kreiranja
    \item \texttt{UpdatedAt}: \texttt{DateTime} - vrijeme posljednje izmjene
\end{itemize}

\subsubsection{Category (Kategorija)}
Entitet koji predstavlja kategoriju transakcija.
\begin{itemize}
    \item \texttt{Id}: \texttt{int} - jedinstveni identifikator
    \item \texttt{Name}: \texttt{string} - naziv kategorije (1-255 karaktera, jedinstven)
    \item \texttt{Description}: \texttt{string?} - opcioni opis kategorije
    \item \texttt{Icon}: \texttt{string?} - opcioni ikona (maksimalno 100 karaktera, npr. emoji)
    \item \texttt{CreatedAt}: \texttt{DateTime} - vrijeme kreiranja
\end{itemize}

\subsubsection{Transaction (Transakcija)}
Entitet koji predstavlja financijsku transakciju (rashod ili prihod).
\begin{itemize}
    \item \texttt{Id}: \texttt{int} - jedinstveni identifikator
    \item \texttt{UserId}: \texttt{int} - ID korisnika (foreign key)
    \item \texttt{TransactionType}: \texttt{TransactionType} - tip transakcije (\texttt{Expense} ili \texttt{Income})
    \item \texttt{Amount}: \texttt{decimal} - apsolutni iznos (uvijek pozitivan)
    \item \texttt{SignedAmount}: \texttt{decimal} - potpisani iznos (negativan za rashode, pozitivan za prihode)
    \item \texttt{Date}: \texttt{DateTime} - datum transakcije
    \item \texttt{Subject}: \texttt{string} - kratak opis transakcije (obavezno, ne može biti prazan string)
    \item \texttt{Notes}: \texttt{string?} - opcioni detaljniji opis
    \item \texttt{PaymentMethod}: \texttt{PaymentMethod} - način plaćanja (Cash, DebitCard, CreditCard, BankTransfer, MobilePayment, PayPal, Crypto, Other)
    \item \texttt{CumulativeDelta}: \texttt{decimal} - kumulativna suma svih \texttt{signed\_amount} vrijednosti do i uključujući ovu transakciju
    \item \texttt{Seq}: \texttt{long} - redni broj za determinističko sortiranje transakcija
    \item \texttt{CategoryId}: \texttt{int?} - ID kategorije (obavezno za rashode, opciono za prihode)
    \item \texttt{TransactionGroupId}: \texttt{int?} - ID grupe transakcija (opcionalno)
    \item \texttt{IncomeSource}: \texttt{string?} - izvor prihoda (npr. "ABC Corporation") - samo za prihode
    \item \texttt{CreatedAt}: \texttt{DateTime} - vrijeme kreiranja
    \item \texttt{UpdatedAt}: \texttt{DateTime} - vrijeme posljednje izmjene
\end{itemize}

\textbf{Napomena}: Balans korisnika se računa kao \texttt{InitialBalance + CumulativeDelta} iz posljednje transakcije.

\subsubsection{TransactionGroup (Grupa transakcija)}
Entitet koji predstavlja grupu povezanih transakcija (npr. putovanje, projekat, vjenčanje).
\begin{itemize}
    \item \texttt{Id}: \texttt{int} - jedinstveni identifikator
    \item \texttt{Name}: \texttt{string} - naziv grupe
    \item \texttt{Description}: \texttt{string?} - opcioni opis grupe
    \item \texttt{UserId}: \texttt{int} - ID korisnika (foreign key)
    \item \texttt{CreatedAt}: \texttt{DateTime} - vrijeme kreiranja
\end{itemize}

\subsection{Rute API-ja}

Sve rute API-ja su verzionisane pod \texttt{/api/v1}. U tabeli ispod su navedene sve dostupne rute sa opisom njihove funkcionalnosti:

\begin{table}[htbp]
\centering
\small
\begin{tabular}{|l|l|p{9cm}|}
\hline
\textbf{HTTP Metoda} & \textbf{Endpoint} & \textbf{Opis} \\
\hline
\multicolumn{3}{|c|}{\textbf{Users (Korisnici)}} \\
\hline
GET & \texttt{/api/v1/users} & Dohvaća listu svih korisnika. Vraća \texttt{GetUsersResponse} sa listom korisnika i ukupnim brojem. \\
\hline
GET & \texttt{/api/v1/users/\{id\}} & Dohvaća korisnika po ID-u. Vraća \texttt{UserResponse} ili 404 ako korisnik ne postoji. \\
\hline
POST & \texttt{/api/v1/users} & Kreira novog korisnika. Zahtijeva \texttt{CreateUserRequest} (name, email, password). Vraća 201 sa kreiranim korisnikom, 400 za validacione greške, ili 409 ako email već postoji. \\
\hline
GET & \texttt{/api/v1/users/\{id\}/balance} & Dohvaća trenutni balans korisnika. Vraća \texttt{UserBalanceResponse} sa initial balance, cumulative delta i current balance. Vraća 404 ako korisnik ne postoji. \\
\hline
PUT & \texttt{/api/v1/users/\{id\}/initial-balance} & Postavlja početni balans korisnika. Zahtijeva \texttt{SetInitialBalanceRequest}. Vraća 200 sa ažuriranim korisnikom ili 404 ako korisnik ne postoji. \\
\hline
\multicolumn{3}{|c|}{\textbf{Categories (Kategorije)}} \\
\hline
GET & \texttt{/api/v1/categories} & Dohvaća listu svih kategorija. Vraća \texttt{GetCategoriesResponse} sa listom kategorija i ukupnim brojem. \\
\hline
GET & \texttt{/api/v1/categories/\{id\}} & Dohvaća kategoriju po ID-u. Vraća \texttt{CategoryResponse} ili 404 ako kategorija ne postoji. \\
\hline
POST & \texttt{/api/v1/categories} & Kreira novu kategoriju. Zahtijeva \texttt{CreateCategoryRequest} (name, description, icon). Naziv mora biti jedinstven. Vraća 201 sa kreiranom kategorijom, 400 za validacione greške, ili 409 ako naziv već postoji. \\
\hline
PUT & \texttt{/api/v1/categories/\{id\}} & Ažurira postojeću kategoriju. Zahtijeva \texttt{UpdateCategoryRequest}. Vraća 200 sa ažuriranom kategorijom, 400 za validacione greške, 404 ako kategorija ne postoji, ili 409 ako novi naziv već postoji. \\
\hline
DELETE & \texttt{/api/v1/categories/\{id\}} & Briše kategoriju. Vraća 204 ako je uspješno obrisana, 404 ako kategorija ne postoji, ili 409 ako je kategorija referencirana u transakcijama. \\
\hline
\multicolumn{3}{|c|}{\textbf{Transactions (Transakcije)}} \\
\hline
GET & \texttt{/api/v1/transactions} & Dohvaća sve transakcije (admin/debug svrhe). Vraća \texttt{GetTransactionsResponse} sa listom transakcija, ukupnim brojem i sažetkom statistika. \\
\hline
GET & \texttt{/api/v1/transactions/\{id\}} & Dohvaća transakciju po ID-u. Vraća \texttt{TransactionResponse} ili 404 ako transakcija ne postoji. \\
\hline
GET & \texttt{/api/v1/transactions/user/\{userId\}} & Dohvaća sve transakcije određenog korisnika. Vraća \texttt{GetTransactionsResponse} ili 404 ako korisnik ne postoji. \\
\hline
GET & \texttt{/api/v1/transactions/user/\{userId\}/type/\{type\}} & Dohvaća transakcije korisnika filtrirane po tipu (EXPENSE ili INCOME). Vraća \texttt{GetTransactionsResponse} ili 400 za nevažeći tip, ili 404 ako korisnik ne postoji. \\
\hline
POST & \texttt{/api/v1/transactions} & Kreira novu transakciju. Zahtijeva \texttt{CreateTransactionRequest} sa svim podacima transakcije. Validira postojanje korisnika, kategorije (ako je navedena) i grupe transakcija (ako je navedena). Subject ne može biti prazan string. Vraća 201 sa kreiranom transakcijom, 400 za validacione greške, ili 404 ako referencirani entiteti ne postoje. \\
\hline
PUT & \texttt{/api/v1/transactions/\{id\}} & Ažurira postojeću transakciju. Zahtijeva \texttt{UpdateTransactionRequest}. Validira postojanje kategorije i grupe transakcija (ako su navedene). Subject ne može biti prazan string. Vraća 200 sa ažuriranom transakcijom, 400 za validacione greške, ili 404 ako transakcija ili referencirani entiteti ne postoje. \\
\hline
DELETE & \texttt{/api/v1/transactions/\{id\}} & Briše transakciju i automatski ažurira cumulative delta za sve naredne transakcije korisnika. Vraća 204 ako je uspješno obrisana ili 404 ako transakcija ne postoji. \\
\hline
\multicolumn{3}{|c|}{\textbf{Transaction Groups (Grupe transakcija)}} \\
\hline
GET & \texttt{/api/v1/transaction-groups} & Dohvaća sve grupe transakcija. Vraća \texttt{GetTransactionGroupsResponse} sa listom grupa i ukupnim brojem. \\
\hline
GET & \texttt{/api/v1/transaction-groups/\{id\}} & Dohvaća grupu transakcija po ID-u. Vraća \texttt{TransactionGroupResponse} ili 404 ako grupa ne postoji. \\
\hline
GET & \texttt{/api/v1/transaction-groups/user/\{userId\}} & Dohvaća sve grupe transakcija određenog korisnika. Vraća listu \texttt{TransactionGroupResponse} ili 404 ako korisnik ne postoji. \\
\hline
POST & \texttt{/api/v1/transaction-groups} & Kreira novu grupu transakcija. Zahtijeva \texttt{CreateTransactionGroupRequest} (name, description, userId). Validira postojanje korisnika. Vraća 201 sa kreiranom grupom, 400 za validacione greške, 422 ako korisnik ne postoji, ili 500 za neočekivane greške. \\
\hline
PUT & \texttt{/api/v1/transaction-groups/\{id\}} & Ažurira postojeću grupu transakcija. Zahtijeva \texttt{UpdateTransactionGroupRequest} (name, description). Vraća 200 sa ažuriranom grupom ili 404 ako grupa ne postoji. \\
\hline
DELETE & \texttt{/api/v1/transaction-groups/\{id\}} & Briše grupu transakcija. Vraća 204 ako je uspješno obrisana ili 404 ako grupa ne postoji. \\
\hline
\multicolumn{3}{|c|}{\textbf{Health Check}} \\
\hline
GET & \texttt{/health} & Provjerava zdravlje aplikacije. Vraća status aplikacije. \\
\hline
\end{tabular}
\caption{Pregled svih ruta Expense Tracker API-ja}
\label{tab:api-routes}
\end{table}

\subsection{Postman kolekcija}

Testovi u Postman kolekciji su organizovani po resursima u sljedećim folderima:

\begin{itemize}
    \item \textbf{Users} - testovi za sve korisničke endpoint-e (GET, POST, PUT za balance)
    \item \textbf{Categories} - testovi za CRUD operacije nad kategorijama
    \item \textbf{Transactions} - testovi za sve transakcije endpoint-e (uključujući filtriranje po korisniku i tipu)
    \item \textbf{Transaction Groups} - testovi za CRUD operacije nad grupama transakcija
    \item \textbf{Health Check} - testovi za provjeru zdravlja aplikacije
\end{itemize}

Svaki folder sadrži testove koji pokrivaju pozitivne i negativne scenarije, uključujući validacione greške, nepostojeće resurse i konflikte.

\subsection{Testni slučajevi}

Postman kolekcija sadrži sljedeće glavne testne slučajeve koji pokrivaju sve funkcionalnosti aplikacije:

\subsubsection{Users testovi}
\begin{itemize}
    \item \textbf{GET All Users} - Testira dohvaćanje liste svih korisnika. Očekivani rezultat: 200 OK sa listom korisnika.
    \item \textbf{GET User by ID} - Testira dohvaćanje korisnika po ID-u. Očekivani rezultat: 200 OK sa korisnikom ili 404 Not Found.
    \item \textbf{POST Create User} - Testira kreiranje novog korisnika. Očekivani rezultat: 201 Created sa kreiranim korisnikom, 400 Bad Request za validacione greške, ili 409 Conflict za duplikat email-a.
    \item \textbf{GET User Balance} - Testira dohvaćanje balansa korisnika. Očekivani rezultat: 200 OK sa balance informacijama ili 404 Not Found.
    \item \textbf{PUT Set Initial Balance} - Testira postavljanje početnog balansa. Očekivani rezultat: 200 OK sa ažuriranim korisnikom ili 404 Not Found.
\end{itemize}

\subsubsection{Categories testovi}
\begin{itemize}
    \item \textbf{GET All Categories} - Testira dohvaćanje liste svih kategorija. Očekivani rezultat: 200 OK sa listom kategorija.
    \item \textbf{GET Category by ID} - Testira dohvaćanje kategorije po ID-u. Očekivani rezultat: 200 OK sa kategorijom ili 404 Not Found.
    \item \textbf{POST Create Category} - Testira kreiranje nove kategorije. Očekivani rezultat: 201 Created sa kreiranom kategorijom, 400 Bad Request za validacione greške, ili 409 Conflict za duplikat naziva.
    \item \textbf{PUT Update Category} - Testira ažuriranje kategorije. Očekivani rezultat: 200 OK sa ažuriranom kategorijom, 404 Not Found, ili 409 Conflict za duplikat naziva.
    \item \textbf{DELETE Category} - Testira brisanje kategorije. Očekivani rezultat: 204 No Content, 404 Not Found, ili 409 Conflict ako je kategorija referencirana.
\end{itemize}

\subsubsection{Transactions testovi}
\begin{itemize}
    \item \textbf{GET All Transactions} - Testira dohvaćanje svih transakcija. Očekivani rezultat: 200 OK sa listom transakcija i sažetkom.
    \item \textbf{GET Transaction by ID} - Testira dohvaćanje transakcije po ID-u. Očekivani rezultat: 200 OK sa transakcijom ili 404 Not Found.
    \item \textbf{GET Transactions by User} - Testira dohvaćanje transakcija određenog korisnika. Očekivani rezultat: 200 OK sa listom transakcija ili 404 Not Found ako korisnik ne postoji.
    \item \textbf{GET Transactions by User and Type} - Testira filtriranje transakcija po tipu. Očekivani rezultat: 200 OK sa filtriranim transakcijama, 400 Bad Request za nevažeći tip, ili 404 Not Found.
    \item \textbf{POST Create Transaction} - Testira kreiranje nove transakcije. Očekivani rezultat: 201 Created sa kreiranom transakcijom, 400 Bad Request za validacione greške (npr. prazan subject), ili 404 Not Found ako referencirani entiteti ne postoje.
    \item \textbf{PUT Update Transaction} - Testira ažuriranje transakcije. Očekivani rezultat: 200 OK sa ažuriranom transakcijom, 400 Bad Request, ili 404 Not Found.
    \item \textbf{DELETE Transaction} - Testira brisanje transakcije. Očekivani rezultat: 204 No Content ili 404 Not Found.
\end{itemize}

\subsubsection{Transaction Groups testovi}
\begin{itemize}
    \item \textbf{GET All Transaction Groups} - Testira dohvaćanje svih grupa transakcija. Očekivani rezultat: 200 OK sa listom grupa.
    \item \textbf{GET Transaction Group by ID} - Testira dohvaćanje grupe po ID-u. Očekivani rezultat: 200 OK sa grupom ili 404 Not Found.
    \item \textbf{GET Transaction Groups by User} - Testira dohvaćanje grupa određenog korisnika. Očekivani rezultat: 200 OK sa listom grupa ili 404 Not Found.
    \item \textbf{POST Create Transaction Group} - Testira kreiranje nove grupe. Očekivani rezultat: 201 Created sa kreiranom grupom, 400 Bad Request, 422 Unprocessable Entity ako korisnik ne postoji, ili 500 Internal Server Error.
    \item \textbf{PUT Update Transaction Group} - Testira ažuriranje grupe. Očekivani rezultat: 200 OK sa ažuriranom grupom ili 404 Not Found.
    \item \textbf{DELETE Transaction Group} - Testira brisanje grupe. Očekivani rezultat: 204 No Content ili 404 Not Found.
\end{itemize}

\subsection{Link/komanda za izvršavanje kolekcije}

Postman kolekcija se može izvršiti na sljedeće načine:

\begin{itemize}
    \item \textbf{Postman aplikacija}: Učitati \texttt{collection.json} fajl u Postman aplikaciju i pokrenuti kolekciju kroz GUI interfejs.
    \item \textbf{Newman CLI}: Koristiti Newman komandu za izvršavanje kolekcije iz terminala:
    \begin{verbatim}
    newman run collection.json
    \end{verbatim}
    \item \textbf{CI/CD integracija}: Newman se može integrisati u CI/CD pipeline za automatsko testiranje API-ja.
\end{itemize}

\textbf{Napomena}: Prije izvršavanja kolekcije, potrebno je konfigurisati environment varijable u Postman-u (npr. base URL API-ja).

